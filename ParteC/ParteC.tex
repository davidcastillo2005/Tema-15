\documentclass[a4paper,12pt]{article}
\usepackage[utf8]{inputenc}
\usepackage[spanish]{babel}
\usepackage{amsmath, amssymb, amsthm}
\usepackage{graphicx}
\usepackage{geometry}
\usepackage{pgfplots}
\usepackage{enumitem} % Paquete clave para personalizar listas
\pgfplotsset{compat=1.17}

% Ajuste de márgenes similar a la hoja de la imagen
\geometry{left=2cm, right=2cm, top=2cm, bottom=2cm}

\begin{document}
	
	% Simulación del encabezado estilo "Tema 15"
	\noindent $\bullet$ \textbf{Tema 15: Aceleración gravitacional variable (Solucionario)}
	
	\noindent Resumen de la aplicación de las EDO en Aceleración gravitacional variable (Solución al ejercicio propuesto: Plano de Fase y Estabilidad).
	
	\begin{itemize}[label=-, leftmargin=0.5cm, itemsep=10pt]
		
		\item \textbf{Parte C (Plano de fase y estabilidad).} Considere el sistema para la altura $y(t)$ y la velocidad $v(t)$:
		\begin{equation}
			\begin{cases}
				\dfrac{dy}{dt} = v \\[10pt]
				\dfrac{dv}{dt} = -\dfrac{100}{(y+1)^2}
			\end{cases}
			\label{eq:sistema}
		\end{equation}
		
		\begin{enumerate}[label=\textbf{\arabic*.}, leftmargin=0.8cm, itemsep=15pt]
			
			% Punto 1
			\item \textbf{Cálculo y clasificación de puntos críticos}
			
			Para encontrar los puntos críticos (o puntos de equilibrio), debemos igualar las derivadas temporales a cero. Buscamos valores $(y^*, v^*)$ tales que:
			
			$$
			\begin{cases}
				v = 0 \\
				-\dfrac{100}{(y+1)^2} = 0
			\end{cases}
			$$
			
			De la primera ecuación obtenemos $v = 0$. Sin embargo, al analizar la segunda ecuación:
			$$ -\frac{100}{(y+1)^2} = 0 $$
			Observamos que el numerador es $-100$ (una constante distinta de cero). Para que una fracción sea cero, su numerador debe ser cero. Como $-100 \neq 0$, la segunda ecuación \textbf{no tiene solución} para ningún valor finito de $y$.
			
			Por tanto, el sistema \textbf{no tiene puntos críticos}.
			
			% Punto 2
			\item \textbf{Construcción del plano de fase e interpretación}
			
			Para construir las trayectorias en el plano de fase $(y, v)$, eliminamos el tiempo $dt$ usando la regla de la cadena:
			$$ \frac{dv}{dy} = \frac{dv/dt}{dy/dt} = \frac{-\frac{100}{(y+1)^2}}{v} $$
			
			Separamos variables e integramos:
			$$ v \, dv = -100 (y+1)^{-2} \, dy $$
			$$ \int v \, dv = \int -100 (y+1)^{-2} \, dy $$
			$$ \frac{1}{2}v^2 = \frac{100}{y+1} + C $$
			
			Las trayectorias son curvas de la forma:
			$$ v = \pm \sqrt{2C + \frac{200}{y+1}} $$
			
			\vspace{0.5cm}
			
			\begin{center}
				\begin{tikzpicture}
					\begin{axis}[
						title={\textbf{Plano de Fase $(y, v)$}},
						xlabel={$y$},
						ylabel={$v$},
						xmin=0, xmax=10,
						ymin=-10, ymax=10,
						axis lines=center,
						grid=major,
						domain=0:10,
						samples=100,
						view={0}{90},
						width=10cm, height=8cm,
						scale only axis
						]
						% Trayectorias para diferentes Energías (C)
						
						% C = -5 (Energía negativa, órbita cerrada/caída)
						\addplot[blue, thick] ({x}, {sqrt(max(0, 2*(-5 + 100/(x+1))))});
						\addplot[blue, thick] ({x}, {-sqrt(max(0, 2*(-5 + 100/(x+1))))});
						
						% C = 0 (Velocidad de escape justa)
						\addplot[red, thick] ({x}, {sqrt(200/(x+1))});
						\addplot[red, thick] ({x}, {-sqrt(200/(x+1))});
						
						% C = 10 (Alta energía)
						\addplot[green!60!black, thick] ({x}, {sqrt(2*(10 + 100/(x+1)))});
						\addplot[green!60!black, thick] ({x}, {-sqrt(2*(10 + 100/(x+1)))});
						
						% Flechas de dirección de flujo (manuales para claridad)
						\node[blue] at (axis cs: 2, 6) {$\rightarrow$}; % v > 0
						\node[blue] at (axis cs: 2, -6) {$\leftarrow$}; % v < 0
						\node at (axis cs: 8, 3) {\small $C>0$ (Escape)};
						\node at (axis cs: 1, 8) {\small Caída};
						
					\end{axis}
				\end{tikzpicture}
			\end{center}
			
			\textbf{Interpretación del Movimiento:}
			
			\begin{enumerate}[label=\alph*)]
				\item \textbf{Dinámica de flujo:}
				\begin{itemize}
					\item En el semiplano superior ($v > 0$): $\frac{dy}{dt} > 0$, la altura aumenta (el cuerpo sube), pero la velocidad disminuye (frenado por gravedad).
					\item En el semiplano inferior ($v < 0$): $\frac{dy}{dt} < 0$, la altura disminuye (el cuerpo cae) y la velocidad se hace cada vez más negativa (acelera hacia abajo).
				\end{itemize}
				
				El movimiento se interpreta como una \textbf{caída hacia el centro}. No existe acercamiento estable a órbita, ya que no hay puntos de equilibrio ni barreras centrífugas en este sistema simplificado.
			\end{enumerate}
			
		\end{enumerate}
	\end{itemize}
	
\end{document}