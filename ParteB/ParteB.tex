\documentclass{article}
\usepackage[utf8]{inputenc}
\usepackage[spanish]{babel}
\usepackage{amsmath}
\usepackage{amssymb}
\usepackage[a4paper, margin=2.5cm]{geometry}

% --- CUERPO DEL DOCUMENTO ---
\begin{document}
	\noindent
	\textbf{Parte B (Bifurcación).} Para capturar cambios en la dinámica gravitacional reducida, 
	considere el modelo unidimensional con parámetro $\mu$:
	
	\[
	\frac{dz}{dt} = \mu z - z^3
	\]
	
	\begin{enumerate}
		\item Determine los puntos de equilibrio en función de $\mu$.
		
		
		$\tilde{\mathbf{z}} \in \mathbb{R}$ es un \textbf{punto de equilibrio} para la ecuación diferencial
		
		\[
		\frac{d\mathbf{z}}{dt} = \mathbf{f}(t, \mathbf{z})
		\]
		
		Si $\mathbf{f}(t, \tilde{\mathbf{z}}) = \mathbf{0}$ para cualquier $t$.
		
		Hallar los puntos que cumplen que la pendiente es igual a 0.
		
		\[
		\mathbf{f}(t, \tilde{\mathbf{z}}) = 0
		\]
		
		\[
		\mu \tilde{\mathbf{z}} - \tilde{\mathbf{z}}^3 = 0
		\]
		
		\[
		\tilde{\mathbf{z}}(\mu - \tilde{\mathbf{z}}^2) = 0
		\]
		
		\[
		\tilde{\mathbf{z}}(\sqrt{\mu} + \tilde{\mathbf{z}})(\sqrt{\mu} - \tilde{\mathbf{z}}) = 0
		\]
		
		\[
		\tilde{\mathbf{z}} = 0 \lor \tilde{\mathbf{z}} = \sqrt{\mu} \lor \tilde{\mathbf{z}} = -\sqrt{\mu}
		\]
		
		Por tanto, si $z = 0 \lor z = \sqrt{\mu} \lor z = -\sqrt{\mu}$ entonces $z$ es un punto de equilibrio para $\frac{dz}{dt} = \mu z - z^3$
		
		\item Clasifique la estabilidad mediante $z' = \mu - 3z^2$.
		
		\item Construya el diagrama de bifurcación en el plano $(\mu, z)$ e identifique el tipo de bifurcación. 
		Interprete qué significa físicamente el paso de $\mu < 0$ a $\mu > 0$ en términos de escape 
		gravitacional frente a captura.
	\end{enumerate}
	
	\noindent
	\textbf{1.} 
\end{document}