\documentclass{article}
\usepackage[utf8]{inputenc}
\usepackage[spanish]{babel}
\usepackage{amsmath}
\usepackage{amssymb}
\usepackage[a4paper, margin=2.5cm]{geometry}

% --- CUERPO DEL DOCUMENTO ---
\begin{document}
	\noindent
	\textbf{Parte B (Bifurcación).} Para capturar cambios en la dinámica gravitacional reducida, 
	considere el modelo unidimensional con parámetro $\mu$:
	
	\[
	\frac{dz}{dt} = \mu z - z^3
	\]
	
	\begin{enumerate}
		\item Determine los puntos de equilibrio en función de $\mu$.
		
		
		$\tilde{\mathbf{z}} \in \mathbb{R}$ es un \textbf{punto de equilibrio} para la ecuación diferencial
		
		\[
		\frac{d{z}}{dt} = {f}(t, {z})
		\]
		
		Si ${f}(t, \tilde{{z}}) = {0}$ para cualquier $t$.
		
		Hallar los puntos que cumplen que la pendiente es igual a 0.
		
		\[
		{f}(t, \tilde{{z}}) = 0
		\]
		
		\[
		\mu \tilde{{z}} - \tilde{{z}}^3 = 0
		\]
		
		\[
		\tilde{{z}}(\mu - \tilde{{z}}^2) = 0
		\]
		
		\[
		\tilde{{z}}(\sqrt{\mu} + \tilde{{z}})(\sqrt{\mu} - \tilde{{z}}) = 0
		\]
		
		\[
		\tilde{{z}} = 0 \lor \tilde{{z}} = \sqrt{\mu} \lor \tilde{{z}} = -\sqrt{\mu}
		\]
		
		Por tanto, si $z = \mathbf{0}$, $z = \mathbf{\sqrt{\mu}}$ ó $z = \mathbf{-\sqrt{\mu}}$ entonces $z$ es un punto de equilibrio para $\frac{dz}{dt} = \mu z - z^3$
		
		\item Clasifique la estabilidad mediante $\dot{z} = \mu - 3z^2$.
		
		\textbf{Teorema} (Estabilidad por linealización).
		Sea el sistema no lineal autónomo unidimensional $\dot{z} = f(z)$, y sea $\tilde{z}$ un punto de equilibrio.
		
		La estabilidad de $\tilde{z}$ se determina analizando el signo de la derivada $f'(\tilde{z})$:
		
		\begin{itemize}
			\item Si la derivada es positiva ($f'(\tilde{z}) > 0$), entonces el punto de 
			equilibrio $\tilde{z}$ es \textbf{inestable} (un repulsor).
			
			\item Si la derivada es negativa ($f'(\tilde{z}) < 0$), entonces el punto de 
			equilibrio $\tilde{z}$ es \textbf{asintóticamente estable} (un atractor).
			
			\item Si la derivada es cero ($f'(\tilde{z}) = 0$), la linealización no decide la 
			estabilidad y se requieren métodos de orden superior. Este es el caso de una 
			posible bifurcación.
		\end{itemize}
		
		
		\begin{itemize}
			\item $\mu > 0$
			\begin{itemize}
				\item $z = 0$
				\[
				\dot{z}(0) = \mu - 3 (0)^2 = \mu
				\]
				
				\[
				\dot{z}(0) = \mu > 0
				\]
				
				Si $\mu > 0$ entonces el punto de equilibrio $\tilde{z} = 0$ es \textbf{inestable}.
				
				\item $z = \sqrt{\mu}$
				\[
				\dot{z}(\sqrt{\mu}) = \mu - 3 (\sqrt{\mu})^2 = \mu - 3\mu = -2\mu
				\]
				
				\[
				\dot{z}(\sqrt{\mu}) = -2\mu < 0
				\]
				
				Si $\mu > 0$ entonces el punto de equilibrio $\tilde{z} = \sqrt{\mu}$ es \textbf{estable}.
				
				\item $z = -\sqrt{\mu}$
				\[
				\dot{z}(-\sqrt{\mu}) = \mu - 3 (-\sqrt{\mu})^2 = \mu - 3\mu = -2\mu
				\]
				
				\[
				\dot{z}(-\sqrt{\mu}) = -2\mu < 0
				\]
				
				Si $\mu > 0$ entonces el punto de equilibrio $\tilde{z} = -\sqrt{\mu}$ es \textbf{estable}.
			\end{itemize}
			
			\item $\mu < 0$
			\begin{itemize}
				\item $z = 0$
				\[
				\dot{z}(0) = \mu < 0
				\]
				
				Si $\mu > 0$ entonces el punto de equilibrio $\tilde{z} = 0$ es \textbf{estable}.
				
				\item $z = \sqrt{\mu}$
				\[
				\dot{z}(\sqrt{\mu}) = -2\mu > 0
				\]
				
				Si $\mu > 0$ entonces el punto de equilibrio $\tilde{z} = \sqrt{\mu}$ es \textbf{inestable}.
				
				\item $z = -\sqrt{\mu}$
				\[
				\dot{z}(-\sqrt{\mu}) = -2\mu > 0
				\]
				
				Si $\mu > 0$ entonces el punto de equilibrio $\tilde{z} = -\sqrt{\mu}$ es \textbf{estable}.
				
			\end{itemize}
			
			\item $\mu = 0$
			\[
			\dot{z}(0) = 0
			\]
			
			\[
			\dot{z}(\sqrt{\mu}) = -2\mu = 0
			\]
			
			\[
			\dot{z}(-\sqrt{\mu}) = 2\mu = 0
			\]
			
			Si $\mu = 0$ entonces el teorema no decide la estabilidad en los puntos de equilibrios.
			
			Analizar gráficamente el comportamiento de $\frac{dz}{dt} = \mu z - z^3$.
			
			\[
			\frac{dz}{dt} = \mu z - z^3 = -z^3
			\]
			
			A continuación se muestra el campo direccional de la función solución $z(t)$ alrededor de $\tilde{z} = 0$.
			
			
			
			Como todas las puntas alrededor del punto $\tilde{z}$ apuntan hacia $\tilde{z}$, las soluciones de $\frac{dz}{dt}$, que comienzan con un punto inicial $(t_0, z_0)$ lo suficientemente cerca de $\tilde{z}$ presentan comportamiento asintótico $\lim_{x \rightarrow +\infty} = \tilde{z}$. 
			
			Por tanto, si $\mu = 0$, entonces $z_0 = 0$, $z_1 = \sqrt{\mu}$ y $z_2 = -\sqrt{\mu}$ son \textbf{asintóticamente estables}.		
			
		\end{itemize}
		
		\item Construya el diagrama de bifurcación en el plano $(\mu, z)$ e identifique el tipo de bifurcación. 
		Interprete qué significa físicamente el paso de $\mu < 0$ a $\mu > 0$ en términos de escape 
		gravitacional frente a captura.
	\end{enumerate}
\end{document}